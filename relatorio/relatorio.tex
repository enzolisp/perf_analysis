%
% sa.tex - exemplo de artigo para o Seminário de Andamento
% $Id: sa.tex,v 1.1.1.1 2005/01/18 23:54:44 avila Exp $
%
% UFRGS TeX Users Group
% Institute of Informatics --- UFRGS
% Porto Alegre, Brazil
% http://www.inf.ufrgs.br/utug
% Discussion list: utug-l@inf.ufrgs.br
%
% Copyright (C) 2001 UFRGS TeX Users Group
% This is free software, distributed under the GNU GPL; please take
% a look in `iiufrgs.cls' to see complete information on using, copying
% and redistributing these files
% Stage 1 - Comp. Sys. Perf Analysis (2025/2) - Lucas M. Schnorr
% Group F: Enzo Lisboa Peixoto - 00584827, Nathan Mattes - 00342941 e Pedro Scholz Soares - 00578266 
%
\documentclass{relatorio}
\usepackage[utf8]{inputenc}
\usepackage{graphicx}

\title{Avaliação de desempenho da equação do calor (Python vs Julia)}
\author{%
Enzo Lisbôa Peixoto\\
Nathan Mattes\\
Pedro Scholz Soares\\
Lucas Mello Schnorr (orientador)
}

\begin{document}
\maketitle

%---------------------------------------------------------------------------
\begin{abstract}
Este documento apresenta as instruções para o formato dos artigos a serem 
incluídos nos Anais da Semana Acadêmica do PPGC da UFRGS, que reúne os 
Seminários de Andamento das Dissertações de Mestrado em elaboração no curso. 
Este Resumo não deve conter menos do que 6~(seis) e mais do que 8~(oito) 
linhas. Ele deve introduzir claramente qual é o tema principal da Dissertação 
de Mestrado e as principais vantagens ou características da metodologia ou 
sistema sendo desenvolvido. A palavra Resumo deve ser escrita em negrito. 
O texto do Resumo deve ser escrito em itálico.
\end{abstract}
%---------------------------------------------------------------------------
\section{DESCRIÇÃO}
\label{sec:intro}
O objeto de estudo deste trabalho é a solução numérica da
equação do calor em uma, duas e três dimensões.
Essa equação descreve a forma como o calor se espalha por um
objeto. Será implementado algoritmos em duas linguagens de
programação distintas, julia e python.
A justificativa das linguagens vem do fato de python ser uma
linguagem consistente para uso científico devido suas
bibliotecas e ser amplamente popular, por outro lado julia
apresenta ótimos resultados em cálculo numérico.

%\[
%\begin{equation}
%\boldsymbol{\frac{\partial u}{\partial t} = \alpha \frac{\partial^2 u}{\partial x^2}, \quad x \in [0,L], \; t > 0} 
%\end{equation}	
%\\[12mm]
%\qquad & \boldsymbol{\frac{\partial u}{\partial t} = \alpha \left( \frac{\partial^2 u}{\partial x^2} + \frac{\partial^2 u}{\partial y^2} \right), \quad (x,y) \in \Omega, \; t > 0} %
	%\\[12mm]    
%\qquad & \boldsymbol{\frac{\partial u}{\partial t} = \alpha \left( \frac{\partial^2 u}{\partial x^2} + \frac{\partial^2 u}{\partial y^2} + \frac{\partial^2 u}{\partial z^2} \right), \quad (x,y,z) \in \Omega, \; t > 0}

%\]


%---------------------------------------------------------------------------
\section{MÉTODO DE ANÁLISE}
Julia utiliza compilação just-in-time (JIT):
A primeira execução de um código em Julia pode ser mais lenta
devido à necessidade de compilar o código antes de executá-lo.
Em execuções subsequentes, o código já está compilado, o que
resulta em tempos de execução significativamente mais rápidos.
Python é uma linguagem interpretada:
Cada vez que um script Python é executado, o interpretador lê
e interpreta o código linha por linha.
Isso pode resultar em tempos de execução mais lentos,
especialmente para tarefas computacionalmente intensivas.
Para garantir uma comparação justa, o tempo de compilação
inicial do Julia será separado do tempo de execução.
O foco da análise será o tempo de execução após a compilação
inicial.
%---------------------------------------------------------------------------
\section{JUSTIFICATIVA}
Resolução de EDP’s já estudada pelos membros do grupo em
outras disciplinas, através da abordagem analítica.
Nenhum membro do grupo explorou a abordagem
computacional da resolução da equação.
Enxergamos esse trabalho como uma oportunidade de estudar
a equação sob outra visão, nos possibilitando abordar
conteúdos relativos a cálculo numérico.

%\begin{figure}
	%\centerline{\includegraphics[width=4cm]{fig}}
	%\caption{Exemplo de utilização de figura no texto.}
	%\label{fig}
%\end{figure}

%---------------------------------------------------------------------------
\section{DEFINIÇÃO DE MÉTRICAS}
Tempo de execução
- Coletar múltiplas amostras para fazer uma média e garantir
robustez nos resultados
- Distinguir o tempo de compilação inicial do Julia do tempo de
execução a fim de ter uma comparação mais justa com Python
Uso de memória
- Avaliar a eficiência da alocação de memória por ambas as
linguagens
%---------------------------------------------------------------------------
\section{CRONOGRAMA PRELIMINAR}
Escrita dos algoritmos de resolução da equação.
Definição dos métodos de medição.
Coleta dos dados.
Análise dos dados.
Escrita do relatório final.
%---------------------------------------------------------------------------
\section*{AGRADECIMENTOS}
Caso o mestrado esteja sendo apoiado por bolsa de estudos de alguma agência 
de fomento ou tenha apoio de alguma universidade/empresa, um agradecimento 
deve ser incluído. Agradecimentos a pessoas que estejam colaborando de 
forma relevante para o trabalho também podem ser incluídos.
%---------------------------------------------------------------------------
\begin{thebibliography}{10}
\bibitem{chu}
Y.~Chu. \underline{Introduction to Computer Organization}. Englewood Cliffs, 
Prentice Hall, 1970.

\bibitem{wagner} F.~Wagner, I.~Jansch-Pôrto, T.S.~Weber e R.F.~Weber. 
Uma proposta de currículo em arquitetura de computadores. In:
\underline{Workshop de Educação em Informática, II}. Caxambu, Agosto 1994. 
Anais. SBC/UFMG, 1994, pp 41-57.

\bibitem{lamport} L.~Lamport. \underline{\LaTeX}: a Document Preparation System. Addison-Wesley, 1994. 2.ed.
\end{thebibliography}
\end{document}
