%
% sa.tex - exemplo de artigo para o Seminário de Andamento
% $Id: sa.tex,v 1.1.1.1 2005/01/18 23:54:44 avila Exp $
%
% UFRGS TeX Users Group
% Institute of Informatics --- UFRGS
% Porto Alegre, Brazil
% http://www.inf.ufrgs.br/utug
% Discussion list: utug-l@inf.ufrgs.br
%
% Copyright (C) 2001 UFRGS TeX Users Group
% This is free software, distributed under the GNU GPL; please take
% a look in `iiufrgs.cls' to see complete information on using, copying
% and redistributing these files
% Stage 1 - Comp. Sys. Perf Analysis (2025/2) - Lucas M. Schnorr
% Group F: Enzo Lisboa Peixoto - 00584827, Nathan Mattes - 00342941 e Pedro Scholz Soares - 00578266 
%
\documentclass{relatorio}
\usepackage[utf8]{inputenc}
\usepackage{graphicx}

\title{Avaliação de desempenho da equação do calor (Python vs Julia)}
\author{%
Enzo Lisbôa Peixoto\\
Nathan Mattes\\
Pedro Scholz Soares\\
Lucas Mello Schnorr (orientador)
}

\begin{document}
\maketitle

%---------------------------------------------------------------------------
\begin{abstract}
Este trabalho tem como objetivo comparar o desempenho das linguages de 
programação Python e Julia na resolução de equações do calor em 1, 2 e 3
dimensões. Se atentando ao tempo percorrido na execução dos programas, assim
como a memória utilizada pelos mesmos, afim de determinar qual das duas é a
mais eficiente para esta aplicação específica. Este trabalho é feito com o
intuito de ser um estudo em cálculo computacional e análise de desempenho.
\end{abstract}
%---------------------------------------------------------------------------
\section{DESCRIÇÃO}
\label{sec:intro}
O objeto de estudo deste trabalho é a solução numérica da equação do calor
em uma, duas e três dimensões. Essa equação descreve a forma como o calor
se espalha por um objeto. Serão implementados algoritmos em duas linguagens
de programação distintas, julia e python. A justificativa vem do fato de
python ser consistente para uso científico devido às suas bibliotecas e ser
amplamente popular. Por outro lado, julia apresenta ótimos resultados em
cálculo numérico.
\\Assim, o objetivo é fazer a análise desses algoritmos e, com isso,
quantificar e compreender os fatores que influenciam no desempenho em cada
linguagem. Para isso, métricas como tempo de execução e uso de memória em
cada dimensão serão analisadas. Ao final do trabalho esperamos compreender
as razões por trás das diferenças observadas, aprimorar nossos conhecimentos
em cálculo numérico e aplicar os conteúdos vistos durante a cadeira de
Análise de desempenho.
%---------------------------------------------------------------------------
\section{MÉTODO DE ANÁLISE}
As duas linguagens possuem abordagens diferentes em relação à execução do 
código. Julia utiliza compilação just-in-time (JIT):
A primeira execução de um código em Julia pode ser mais lenta
devido à necessidade de compilar o código antes de executá-lo.
Em execuções subsequentes, o código já estará compilado, o que
resulta em tempos de execução significativamente mais rápidos.
Por sua vez, Python é uma linguagem interpretada:
Cada vez que um script Python é executado, o interpretador lê
e interpreta o código linha por linha.
Isso pode resultar em tempos de execução mais lentos,
especialmente para tarefas computacionalmente intensivas.
Para garantir uma comparação justa, o tempo de compilação
inicial do Julia será separado do tempo de execução.
O foco da análise será o tempo de execução após a compilação
inicial, assim como o uso de recursos pela aplicação durante
sua execução, tais como memória utilizada.
%---------------------------------------------------------------------------
\section{JUSTIFICATIVA}
A resolução de Equações Diferenciais Parciais já foi estudada pelos membros
do grupo em outras disciplinas, através da abordagem analítica. No entanto,
nenhum de nós explorou a dimensão computacional da solução, que é crucial
para problemas complexos do mundo real. Enxergamos esse trabalho como uma
oportunidade de estudar a equação sob outra visão, possibilitando abordar
conteúdos relativos à cálculo numérico. Também é uma chance de se familiarizar
com Julia, já que nenhum dos integrantes possui experiência com a linguagem. 
%---------------------------------------------------------------------------
\section{DEFINIÇÃO DE MÉTRICAS}
A análise de desempenho deste trabalho será fundamentada em duas métricas
principais: o tempo de execução e o uso de memória. Para obter resultados
robustos, faremos múltiplas medições para cada implementação. Depois, será 
calculada uma média para garantir robustez nos resultados. Julia, por 
utilizar a compilação just-in-time (JIT), terá o tempo inicial de compilação
separado do tempo de execução real do código. Essa abordagem nos permitirá
comparar o desempenho de Julia com o de Python, garantindo uma avaliação
mais justa. Paralelamente, monitoraremos o uso de memória para avaliar a
eficiência de alocação de ambas as linguagens, uma métrica crucial para
entender o quão viável cada solução é para problemas de grande escala.
%---------------------------------------------------------------------------
\section{CRONOGRAMA PRELIMINAR}
\begin{tabular}{l r}
Atividade & Prazo Máximo\\
Escrita dos algoritmos de resolução da equação & 05/09\\
Definição dos métodos de coleta & 12/09\\
Coleta parcial dos dados & 26/09\\
Análise dos dados parciais & 04/10\\
Apresentação dos dados parciais & 06/10\\
Coleta total dos dados & 24/10\\
Análise dos dados & 07/11\\
Escrita do relatório final &  30/11\\
\end{tabular}
%---------------------------------------------------------------------------
\end{document}
