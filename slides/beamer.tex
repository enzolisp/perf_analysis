%% example.tex
%% Copyright 2012 Bruno Menegola
%
% This work may be distributed and/or modified under the
% conditions of the LaTeX Project Public License, either version 1.3
% of this license or (at your option) any later version.
% The latest version of this license is in
%   http://www.latex-project.org/lppl.txt
% and version 1.3 or later is part of all distributions of LaTeX
% version 2005/12/01 or later.
%
% This work has the LPPL maintenance status ‘maintained’.
%
% The Current Maintainer of this work is Bruno Menegola.
%
% This work consists of all files listed in MANIFEST
%
%
% Description
% ===========
%
% This is an example latex document to build presentation slides based on
% the beamer class using the Inf theme.

\documentclass{beamer}

\usepackage[T1]{fontenc}
\usepackage[brazil]{babel}
\usepackage[utf8]{inputenc}

% Choose the Inf theme
\usetheme{Inf}

% Define the title with \title[short title]{long title}
% Short title is optional
\title[Etapa 1: Proposta]
      {Avaliação de Desempenho Etapa 1: Proposta}

% Optional subtitle
% \subtitle{Congresso XYZ}

\date{Agosto de 2025}

% Author information
\author{Enzo Lisbôa Peixoto, Nathan Mattes e \\Pedro Scholz Soares}
\institute{Instituto de Informática --- UFRGS\\\texttt{inf.ufrgs.br/\~{}bmenegola}}

\begin{document}

% Command to create title page
\InfTitlePage

%\begin{frame}
%  \frametitle{Agenda}
%  \tableofcontents
%\end{frame}

\section{a}

\frame{
    \frametitle{Descrição}

    \begin{itemize}
        \item O objeto de estudo deste trabalho é a solução numérica da equação do calor em uma, duas e três dimensões. 
		\item Essa equação descreve a forma como o calor se espalha por um objeto. Será implementado algoritmos em duas linguagens de programação distintas, julia e python. 
        \item A justificativa das linguagens vem do fato de python ser uma linguagem consistente para uso científico devido suas bibliotecas e ser amplamente popular, por outro lado julia apresenta ótimos resultados em cálculo numérico.
    \end{itemize}
}

\section{a}

\frame{
    \frametitle{Método de Análise}

    \begin{itemize}
        \item Usaremos medição como método de análise.
    \end{itemize}
}

%\section{Outra seção}
%\subsection{Primeira subseção}

\section{a}

\frame{
    \frametitle{Justificativa}

    \begin{itemize}
        \item Uma temática já estudada pelos membros do grupo em outras disciplinas.
        \item Enxergamos esse trabalho como uma oportunidade de estudar a equação sob outra visão, nos possibilitando abordar conteúdos relativos a cálculo numérico.
        \item Nenhum membro do grupo explorou a abordagem computacional da resolução da equação.
        %\begin{itemize}
        %    \item nome da seção
        %    \item subseção
        %    \item título do slide
        %\end{itemize}
    \end{itemize}
}

\section{a}

%\section{Blocos}

%\begin{frame}[plain]
% \sectionpage
%\end{frame}

\frame{
    \frametitle{Definição das Métricas}

    \begin{theorem}
        foo
    \end{theorem}
    \begin{proof}
        bar
    \end{proof}
}

\frame{
    \frametitle{Cronograma Preliminar}

    \begin{exampleblock}{exemplo}
        foo
    \end{exampleblock}
}

%\frame{
%    \frametitle{Alertas}

%    \begin{alertblock}{alerta}
%        foo
%    \end{alertblock}
%}

\section*{}

\begin{frame}
    \frametitle{Obrigado!}
    \InfContacts
\end{frame}

\end{document}