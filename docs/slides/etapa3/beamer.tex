% Stage 3 - Comp. Sys. Perf Analysis (2025/2) - Lucas M. Schnorr
% Group F: Enzo Lisboa Peixoto - 00584827, Nathan Mattes - 00342941 e Pedro Scholz Soares - 00578266 
%

\documentclass{beamer}

\usepackage[T1]{fontenc}
\usepackage[brazil]{babel}
\usepackage[utf8]{inputenc}
\usepackage{graphicx}
\usepackage{pdfpages}
\usepackage[section]{placeins}

% Choose the Inf theme
\usetheme{Inf}

% Define the title with \title[short title]{long title}
% Short title is optional
\title[Etapa 3: Parcial]
      {Avaliação de Desempenho Etapa 3: Final}

\date{Outubro de 2025}

% Author information
\author{Enzo Lisbôa Peixoto, Nathan Mattes e \\Pedro Scholz Soares}
\institute{Instituto de Informática --- UFRGS\\\texttt{elpeixoto@inf.ufrgs.br\\nmattes@inf.ufrgs.br\\pedro.soares@inf.ufrgs.br\\}}

\begin{document}

% Command to create title page
\InfTitlePage

% 1D Heat equation


\section{Detalhamento}

\frame{
    \frametitle{Descrição}

    \begin{itemize}
        \item Docker que carrega as dependências das linguagens Julia e Python. 
		\item randexp.R gera testes em ordem aleatória. Combinações entre (Julia
         ou Python), (1 dimensão, 2 dimensões e 3 dimensões) e (low, mid e high).
        \item Cada combinação é executada 10 vezes.
        \item runExperiment.sh executa os códigos em Julia e Python com as entradas
        já definidas, coleta dados e de memória e tempo de execução e salva os 
        resultados.
        \item Os resultados são tratados por scripts R e com eles são gerados gráficos.
    \end{itemize}
}


\frame{
    \frametitle{Descrição}
    \begin{figure}[!htb]
    \centering
        \includegraphics[width=0.8\textwidth]{../../imagensEtapa2/OrgranogramaExperimentos.jpeg}
        \caption{Organograma dos Experimentos}
    \end{figure}
}

\section{Detalhamento}

\frame{
    \frametitle{Método de Coleta de Dados}

    \begin{itemize}
        \item São coletados o uso de memória e o tempo de execução dos algoritmos.
        \item Uso de memória é coletado pelo Docker.
        \item Tempo de execução é calculado pelos algoritmos durante a execução.
    \end{itemize}
}

\section{Detalhamento}

\frame{
    \frametitle{Resultados Finais}

    \begin{itemize}
        \item 12th Gen Intel(R) Core(TM) i3-1215U e 8GB de ram.
    \end{itemize}
    \begin{figure}[!htb]
    \centering
    \includegraphics[width=0.75\textwidth, height=0.60\textheight]{../../imagensEtapa2/boxplotJulia.pdf}
    \caption{Tempo de execução do Julia para todos os casos}
    \end{figure}
}

\frame{
    \frametitle{Resultados Finais}
    \begin{figure}[!htb]
    \centering
    \includegraphics[width=0.8\textwidth, height=0.65\textheight]{../../imagensEtapa2/boxplotPython.pdf}
    \caption{Tempo de execução do Python para todos os casos}
    \end{figure}
}

\frame{
    \frametitle{Resultados Finais}
    \begin{figure}[!htb]
    \centering
    \includegraphics[width=0.8\textwidth, height=0.65\textheight]{../../imagensEtapa2/Rplots.pdf}
    \caption{Comparação do tempo de execução entre Python e Julia para todos os casos}
    \end{figure}
}

\frame{
    \frametitle{Resultados Finais}
    \begin{figure}[!htb]
    \centering
    \includegraphics[width=0.8\textwidth, height=0.65\textheight]{../../imagensEtapa2/mem_bars_1d.png}
    \caption{Uso de memória para 1 dimensão}
    \end{figure}
}

\frame{
    \frametitle{Resultados Finais}
    \begin{figure}[!htb]
    \centering
    \includegraphics[width=0.8\textwidth, height=0.65\textheight]{../../imagensEtapa2/mem_bars_2d.png}
    \caption{Uso de memória para 2 dimensões}
    \end{figure}
}

\section{Detalhamento}

\frame{
    \frametitle{Discussão dos Resultados}

}

\frame{
    \frametitle{Discussão dos Resultados}

}

\section{Detalhamento}

\frame{
    \frametitle{Conclusões e Melhorias}
    
}

\section*{}

\begin{frame}
    \frametitle{Obrigado!}
    \InfContacts
\end{frame}

\end{document}
