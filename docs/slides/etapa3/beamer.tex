% Stage 3 - Comp. Sys. Perf Analysis (2025/2) - Lucas M. Schnorr
% Group F: Enzo Lisboa Peixoto - 00584827, Nathan Mattes - 00342941 e Pedro Scholz Soares - 00578266 

\documentclass{beamer}

\usepackage[T1]{fontenc}
\usepackage[brazil]{babel}
\usepackage[utf8]{inputenc}
\usepackage{graphicx}
\usepackage{pdfpages}
\usepackage[section]{placeins}
\usepackage{booktabs}

% Choose the Inf theme
\usetheme{Inf}

% Define the title with \title[short title]{long title}
% Short title is optional
\title[Etapa 3: Final]
      {Avaliação de Desempenho Etapa 3: Final}

\date{Outubro de 2025}

% Author information
\author{Enzo Lisbôa Peixoto, Nathan Mattes e \\Pedro Scholz Soares}
\institute{Instituto de Informática --- UFRGS\\\texttt{elpeixoto@inf.ufrgs.br\\nmattes@inf.ufrgs.br\\pedro.soares@inf.ufrgs.br\\}}

\begin{document}

% Command to create title page
\InfTitlePage

% 1D Heat equation


\section{Revisão da Proposta e Metodologia
}

\frame{
    \frametitle{Descrição}

    \begin{itemize}
        \item Problema: Resolução da Equação de Calor (Difusão) usando o método das Diferenças Finitas em 1D, 2D e 3D;
		\item Comparação: Linguagem Compilada JIT (Julia) vs. Linguagem Interpretada (Python);
        \item Fatores: Linguagem (2 níveis), Dimensão (3 níveis), Tamanho do Problema (3 níveis: Low, Mid, High);
        \item Métricas: Tempo de Execução e Pico de Memória;
        \item Repetições: 50 execuções por configuração para garantir significância estatística.

    \end{itemize}
}



\section{Descrição do Ambiente, Ferramentas e Métodos}

\frame{
    \frametitle{Controle do experimento}

    \begin{itemize}
        \item CPU Pinning: Uso de --cpuset-cpus="0" para isolar o processo em um único núcleo e evitar trocas de contexto;
        \item Frequency Governor: Fixado em performance para evitar oscilação de clock;
        \item Turbo Boost: Desativado.
    \end{itemize}
}

\frame{
    \frametitle{Reprodutibilidade e Especificações}

    \begin{itemize}
        \item 12th Gen Intel(R) Core(TM) i3-1215U (6/8);
        \item 8GB de Memória
        \item Uso de Docker para garantir reprodutibilidade do ambiente de software;
        \item Script automacao.py que orquestra todo o fluxo;
    \end{itemize}
}

\frame{
    \frametitle{resultados}
    \begin{figure}[!htb]
    \centering
        \includegraphics[width=0.8\textwidth]{../../imagensEtapa2/OrgranogramaExperimentos.jpeg}
        \caption{Organograma dos Experimentos}
    \end{figure}
}
\section{Detalhamento}


\begin{frame}
    \frametitle{Resultados}
    
    \begin{center}
        \textbf{\large Gráficos de Tempo de Execução, carga intermediária - Julia}
    \end{center}
    
    \vspace{0.2cm} 
    
    \begin{columns}[c]
        \begin{column}{0.33\textwidth}
            \centering
            \includegraphics[width=\linewidth]{../../imagensEtapa3/estabilidade_julia_1_mid.pdf}
            \vspace{0.1cm}
            {\tiny \textbf{1D}}
        \end{column}
        
        \begin{column}{0.33\textwidth}
            \centering
            \includegraphics[width=\linewidth]{../../imagensEtapa3/estabilidade_julia_2_mid.pdf}
            \vspace{0.1cm}
            {\tiny \textbf{2D}}
        \end{column}
        
        \begin{column}{0.33\textwidth}
            \centering
            \includegraphics[width=\linewidth]{../../imagensEtapa3/estabilidade_julia_3_mid.pdf}
            \vspace{0.1cm}
            {\tiny \textbf{3D}}
        \end{column}
    \end{columns}
\end{frame}

\begin{frame}
	\frametitle{Resultados}
    
    \begin{center}
        \textbf{\large Gráficos de Tempo de Execução, carga intermediária - Python}
    \end{center}
    
    \vspace{0.2cm} 
    
    \begin{columns}[c]
        \begin{column}{0.33\textwidth}
            \centering
            \includegraphics[width=\linewidth]{../../imagensEtapa3/estabilidade_python_1_mid.pdf}
            \vspace{0.1cm}
            {\tiny \textbf{1D}}
        \end{column}
        
        \begin{column}{0.33\textwidth}
            \centering
            \includegraphics[width=\linewidth]{../../imagensEtapa3/estabilidade_python_2_mid.pdf}
            \vspace{0.1cm}
            {\tiny \textbf{2D}}
        \end{column}
        
        \begin{column}{0.33\textwidth}
            \centering
            \includegraphics[width=\linewidth]{../../imagensEtapa3/estabilidade_python_3_mid.pdf}
            \vspace{0.1cm}
            {\tiny \textbf{3D}}
        \end{column}
    \end{columns}
\end{frame}

\begin{frame}
	\frametitle{Resultados}
    
    \begin{center}
        \textbf{\large Gráficos Memória}
    \end{center}
    
    \vspace{0.2cm} 
    
    \begin{columns}[c]
        \begin{column}{0.33\textwidth}
            \centering
            \includegraphics[width=\linewidth]{../../imagensEtapa3/mem_bars_1d.pdf}
            \vspace{0.1cm}
            {\tiny \textbf{1D}}
        \end{column}
        
        \begin{column}{0.33\textwidth}
            \centering
            \includegraphics[width=\linewidth]{../../imagensEtapa3/mem_bars_2d.pdf}
            \vspace{0.1cm}
            {\tiny \textbf{2D}}
        \end{column}
        
        \begin{column}{0.33\textwidth}
            \centering
            \includegraphics[width=\linewidth]{../../imagensEtapa3/mem_bars_3d.pdf}
            \vspace{0.1cm}
            {\tiny \textbf{3D}}
        \end{column}
    \end{columns}
\end{frame}

\frame{
\frametitle{Resultados}
\begin{table}[]
    \centering
    \caption{Regressão Linear: Tempo de Execução}
    \begin{tabular}{cccc}
        \toprule
        \textbf{Linguagem} & \textbf{Dimensão} & \textbf{$R^2$} & \textbf{$R^2_{adj}$} \\
        \midrule
        Python & 1D & \textbf{0.9631} & 0.9627 \\
        Python & 2D & \textbf{0.9720} & 0.9717 \\
        Python & 3D & \textbf{0.9773} & 0.9771 \\
        \midrule
        Julia & 1D & \textbf{0.9924} & 0.9923 \\
        Julia & 2D & \textbf{0.9760} & 0.9758 \\
        Julia & 3D & \textbf{0.9675} & 0.9671 \\
        \bottomrule
    \end{tabular}
\end{table}
}

\frame{
\frametitle{Resultados}
    \begin{table}[]
    \centering
    \caption{Regressão Linear: Pico de Memória}
    \begin{tabular}{cccc}
        \toprule
        \textbf{Linguagem} & \textbf{Dimensão} & \textbf{$R^2$} & \textbf{$R^2_{adj}$} \\
        \midrule
        Python & 1D & \textbf{1.0000} & 1.0000 \\
        Python & 2D & \textbf{0.9908} & 0.9907 \\
        Python & 3D & \textbf{0.9661} & 0.9657 \\
        \midrule
        Julia & 1D & \textbf{1.0000} & 1.0000 \\
        Julia & 2D & \textbf{0.9908} & 0.9907 \\
        Julia & 3D & \textbf{0.9667} & 0.9663 \\
        \bottomrule
    \end{tabular}
\end{table}
}


\frame{
    \frametitle{resultados}
    \begin{figure}[!htb]
    \centering
        \includegraphics[width=0.8\textwidth]{../../imagensEtapa3/regressao_julia_2d.pdf}
        \caption{Regressão linear julia 2d}
    \end{figure}
}

\begin{frame}
	\frametitle{Resultados}
    
    \begin{center}
        \textbf{\large Regressão Tempo de Execução}
    \end{center}
    
    \vspace{0.2cm} 
    
    \begin{columns}[c]
        \begin{column}{0.5\textwidth}
            \centering
            \includegraphics[width=\linewidth]{../../imagensEtapa3/regressao_julia_1d.pdf}
            \vspace{0.1cm}
            {\tiny \textbf{1D}}
        \end{column}
        
        \begin{column}{0.5\textwidth}
            \centering
            \includegraphics[width=\linewidth]{../../imagensEtapa3/regressao_julia_3d.pdf}
            \vspace{0.1cm}
            {\tiny \textbf{3D}}
        \end{column}
        
    \end{columns}
\end{frame}

%--------------------------

\frame{
    \frametitle{resultados}
    \begin{figure}[!htb]
    \centering
        \includegraphics[width=0.8\textwidth]{../../imagensEtapa3/regressao_python_2d.pdf}
        \caption{Regressão linear python 2d}
    \end{figure}
}

\begin{frame}
	\frametitle{Resultados}
    
    \begin{center}
        \textbf{\large Regressão Tempo de Execução}
    \end{center}
    
    \vspace{0.2cm} 
    
    \begin{columns}[c]
        \begin{column}{0.5\textwidth}
            \centering
            \includegraphics[width=\linewidth]{../../imagensEtapa3/regressao_python_1d.pdf}
            \vspace{0.1cm}
            {\tiny \textbf{1D}}
        \end{column}
        
        \begin{column}{0.5\textwidth}
            \centering
            \includegraphics[width=\linewidth]{../../imagensEtapa3/regressao_python_3d.pdf}
            \vspace{0.1cm}
            {\tiny \textbf{3D}}
        \end{column}
        
    \end{columns}
\end{frame}


%--------------------------

\frame{
    \frametitle{resultados}
    \begin{figure}[!htb]
    \centering
        \includegraphics[width=0.8\textwidth]{../../imagensEtapa3/regressao_peak_mem_julia_2d.pdf}
        \caption{Regressão linear julia 2d}
    \end{figure}
}

\begin{frame}
	\frametitle{Resultados}
    
    \begin{center}
        \textbf{\large Regressão Pico de Memória}
    \end{center}
    
    \vspace{0.2cm} 
    
    \begin{columns}[c]
        \begin{column}{0.5\textwidth}
            \centering
            \includegraphics[width=\linewidth]{../../imagensEtapa3/regressao_peak_mem_julia_1d.pdf}
            \vspace{0.1cm}
            {\tiny \textbf{1D}}
        \end{column}
        
        \begin{column}{0.5\textwidth}
            \centering
            \includegraphics[width=\linewidth]{../../imagensEtapa3/regressao_peak_mem_julia_3d.pdf}
            \vspace{0.1cm}
            {\tiny \textbf{3D}}
        \end{column}
        
    \end{columns}
\end{frame}

%--------------------------

\frame{
    \frametitle{resultados}
    \begin{figure}[!htb]
    \centering
        \includegraphics[width=0.8\textwidth]{../../imagensEtapa3/regressao_peak_mem_python_2d.pdf}
        \caption{Regressão linear python 2d}
    \end{figure}
}

\begin{frame}
	\frametitle{Resultados}
    
    \begin{center}
        \textbf{\large Regressão Pico de Memória}
    \end{center}
    
    \vspace{0.2cm} 
    
    \begin{columns}[c]
        \begin{column}{0.5\textwidth}
            \centering
            \includegraphics[width=\linewidth]{../../imagensEtapa3/regressao_peak_mem_python_1d.pdf}
            \vspace{0.1cm}
            {\tiny \textbf{1D}}
        \end{column}
        
        \begin{column}{0.5\textwidth}
            \centering
            \includegraphics[width=\linewidth]{../../imagensEtapa3/regressao_peak_mem_python_3d.pdf}
            \vspace{0.1cm}
            {\tiny \textbf{3D}}
        \end{column}
        
    \end{columns}
\end{frame}

\section{Discussão dos Resultados}

\begin{frame}
    \frametitle{Discussão: O Trade-off Velocidade vs. Memória}
    \begin{itemize}
        \item \textbf{Desempenho Bruto:} Como esperado, Julia demonstrou ser ordens de magnitude mais rápida que Python em todas as dimensões testadas.
        
        \vspace{0.3cm}
        
        \item \textbf{Contrapartida de Memória:} 
        \begin{itemize}
            \item Os dados mostram que \textbf{Julia consome significativamente mais memória RAM} que Python para realizar a mesma tarefa.
            \item \textbf{Interpretação:} Este é o custo da velocidade. A infraestrutura de compilação \textit{Just-In-Time} (JIT) e o gerenciamento de tipos do Julia exigem uma alocação de recursos muito mais agressiva do que o interpretador leve do Python.
        \end{itemize}
    \end{itemize}
\end{frame}

\begin{frame}
    \frametitle{Discussão: O Gargalo dos Loops Explícitos}
    \begin{itemize}
        \item \textbf{Por que Python foi lento?}
        \begin{itemize}
            \item A implementação utilizou loops explícitos (`for`) iterando sobre a matriz ponto a ponto.
            \item Em Python, cada iteração do loop carrega o \textit{overhead} da tipagem dinâmica e verificação de objetos, o que torna o processamento escalar extremamente ineficiente.
        \end{itemize}
        
        \vspace{0.3cm}
        
        \item \textbf{A Vantagem de Julia:}
        \begin{itemize}
            \item Julia foi projetada para otimizar loops. O compilador infere os tipos e gera código de máquina eficiente para as iterações, eliminando o gargalo que existe no Python puro.
        \end{itemize}
    \end{itemize}
\end{frame}

\begin{frame}
    \frametitle{Discussão: Produtividade e Abstração}
    \begin{itemize}
        \item \textbf{O "Problema das Duas Linguagens":}
        \begin{itemize}
            \item Python resolve a facilidade de escrita, mas exige C/C++ para performance crítica.
            \item Julia resolve ambos: permite escrever código matemático de alto nível (idêntico ao Python) com a performance de linguagens compiladas.
        \end{itemize}
        
        \vspace{0.3cm}
        
        \item \textbf{Conclusão da Análise:}
        \begin{itemize}
            \item Para algoritmos científicos que dependem de iterações complexas, Julia oferece a melhor relação entre \textbf{tempo de desenvolvimento} e \textbf{tempo de execução}, contanto que o maior uso de memória seja aceitável.
        \end{itemize}
    \end{itemize}
\end{frame}

\section{Conclusão}

\begin{frame}
    \frametitle{Conclusões}
    \begin{block}{Conclusão Técnica}
        Para computação científica baseada em iterações explícitas (simulações físicas), Julia é a escolha superior, oferecendo desempenho próximo de C/Fortran com a facilidade de escrita de Python.
    \end{block}
    
    \vspace{0.5cm}
    
   \begin{block}{Síntese dos Resultados}
        O estudo evidenciou que não existe opção perfeita: o ganho de desempenho do Julia (JIT) exige uma alocação de memória agressiva. A escolha da linguagem deve ser baseada no recurso mais escasso do sistema (Tempo ou RAM).
    \end{block}
\end{frame}

\begin{frame}
    \frametitle{Recomendações e Trabalhos Futuros}
    \begin{itemize}
        \item \textbf{Melhoria de Desempenho:} 
        \begin{itemize}
            \item Implementar paralelismo (\texttt{Threads.@threads} em Julia).
        \end{itemize}
        
        \item \textbf{Boas Práticas de Benchmarking:}
        \begin{itemize}
            \item Sempre travar a frequência da CPU (\textit{governor performance}).
            \item Realizar rodadas de aquecimento (\textit{warm-up}) antes da medição.
            \item Usar Docker com \textit{CPU Pinning} para reprodutibilidade.
        \end{itemize}
    \end{itemize}
\end{frame}
\section*{}

\begin{frame}
    \frametitle{Obrigado!}
    \InfContacts
\end{frame}

\end{document}
