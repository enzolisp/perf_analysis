% Stage 1 - Comp. Sys. Perf Analysis (2025/2) - Lucas M. Schnorr
% Group F: Enzo Lisboa Peixoto - 00584827, Nathan Mattes - 00342941 e Pedro Scholz Soares - 00578266 
%

\documentclass{beamer}

\usepackage[T1]{fontenc}
\usepackage[brazil]{babel}
\usepackage[utf8]{inputenc}

% Choose the Inf theme
\usetheme{Inf}

% Define the title with \title[short title]{long title}
% Short title is optional
\title[Etapa 1: Proposta]
      {Avaliação de Desempenho Etapa 1: Proposta}

\date{Agosto de 2025}

% Author information
\author{Enzo Lisbôa Peixoto, Nathan Mattes e \\Pedro Scholz Soares}
\institute{Instituto de Informática --- UFRGS\\\texttt{elpeixoto@inf.ufrgs.br\\nmattes@inf.ufrgs.br\\pedro.soares@inf.ufrgs.br\\}}

\begin{document}

% Command to create title page
\InfTitlePage

% 1D Heat equation


\section{Detalhamento}

\frame{
    \frametitle{Descrição}

    \begin{itemize}
        \item O objeto de estudo deste trabalho é a solução numérica da equação do calor em uma, duas e três dimensões. 
		\item Essa equação descreve a forma como o calor se espalha por um objeto. Será implementado algoritmos em duas linguagens de programação distintas, julia e python. 
        
       % \begin{equation*} \label{eq:heat}
       % \boldsymbol{u_t = \alpha \, u_{xx}}
       % \end{equation*}
        
        \item A justificativa das linguagens vem do fato de python ser uma linguagem consistente para uso científico devido suas bibliotecas e ser amplamente popular, por outro lado julia apresenta ótimos resultados em cálculo numérico.
    \end{itemize}
}


\frame{
    \frametitle{Descrição}
    \[
    \begin{aligned}  
    \qquad & \boldsymbol{\frac{\partial u}{\partial t} = \alpha \frac{\partial^2 u}{\partial x^2}, \quad x \in [0,L], \; t > 0} 
        \\[12mm]
    \qquad & \boldsymbol{\frac{\partial u}{\partial t} = \alpha \left( \frac{\partial^2 u}{\partial x^2} + \frac{\partial^2 u}{\partial y^2} \right), \quad (x,y) \in \Omega, \; t > 0} 
        \\[12mm]    
    \qquad & \boldsymbol{\frac{\partial u}{\partial t} = \alpha \left( \frac{\partial^2 u}{\partial x^2} + \frac{\partial^2 u}{\partial y^2} + \frac{\partial^2 u}{\partial z^2} \right), \quad (x,y,z) \in \Omega, \; t > 0}
    \end{aligned}
    \]
}

\section{Detalhamento}

\frame{
    \frametitle{Método de Análise}

    \begin{itemize}
        \item Usaremos medição como método de análise.
        \item São algoritmos de implementação simples.
        \item Podemos analisar dados obtidos de forma empírica.
    \end{itemize}
}

\frame{
    \frametitle{Método de Análise}
    
    \begin{itemize}
        \item Julia utiliza compilação just-in-time (JIT): 
        \begin{itemize}
            \item A primeira execução de um código em Julia pode ser mais lenta devido à necessidade de compilar o código antes de executá-lo.
            \item Em execuções subsequentes, o código já está compilado, o que resulta em tempos de execução significativamente mais rápidos.
        \end{itemize}
        \item Python é uma linguagem interpretada:
        \begin{itemize}
            \item Cada vez que um script Python é executado, o interpretador lê e interpreta o código linha por linha.
            \item Isso pode resultar em tempos de execução mais lentos, especialmente para tarefas computacionalmente intensivas.
        \end{itemize}
        \item Para garantir uma comparação justa, o tempo de compilação inicial do Julia será separado do tempo de execução.
        \item O foco da análise será o tempo de execução após a compilação inicial.

    \end{itemize}
}

\section{Detalhamento}

\frame{
    \frametitle{Justificativa}

    \begin{itemize}
        \item Resolução de EDP's já estudada pelos membros do grupo em outras disciplinas, através da abordagem analítica.
        \item Nenhum membro do grupo explorou a abordagem computacional da resolução da equação.
        \item Enxergamos esse trabalho como uma oportunidade de estudar a equação sob outra visão, nos possibilitando abordar conteúdos relativos a cálculo numérico.
    \end{itemize}
}

\section{Detalhamento}

\frame{
    \frametitle{Definição das Métricas}

    \begin{itemize}
        \item Tempo de execução
        \begin{itemize}
            \item Coletar múltiplas amostras para fazer uma média e garantir robustez nos resultados
            \item Distinguir o tempo de compilação inicial do Julia do tempo de execução a fim de ter uma comparação mais justa com Python
        \end{itemize}
        \item Uso de memória
        \begin{itemize}
            \item Avaliar a eficiência da alocação de memória por ambas as linguagens
        \end{itemize}
    \end{itemize}
}

\section{Detalhamento}

\frame{
    \frametitle{Cronograma Preliminar}
    \begin{tabular}{l r}
        Atividade & Prazo Máximo\\
        Escrita dos algoritmos de resolução da equação & 05/09\\
        Definição dos métodos de coleta & 12/09\\
        Coleta parcial dos dados & 26/09\\
        Analise dos dados parciais & 04/10\\
        Apresentação dos dados parciais & 06/10\\
        Coleta total dos dados & 24/10\\
        Analise dos dados & 07/11\\
        Escrita do relatório final &  30/11\\
    \end{tabular}
}

\section*{}

\begin{frame}
    \frametitle{Obrigado!}
    \InfContacts
\end{frame}

\end{document}
