% Stage 2 - Comp. Sys. Perf Analysis (2025/2) - Lucas M. Schnorr
% Group F: Enzo Lisboa Peixoto - 00584827, Nathan Mattes - 00342941 e Pedro Scholz Soares - 00578266 
%

\documentclass{beamer}

\usepackage[T1]{fontenc}
\usepackage[brazil]{babel}
\usepackage[utf8]{inputenc}

% Choose the Inf theme
\usetheme{Inf}

% Define the title with \title[short title]{long title}
% Short title is optional
\title[Etapa 2: Parcial]
      {Avaliação de Desempenho Etapa 2: Parcial}

\date{Outubro de 2025}

% Author information
\author{Enzo Lisbôa Peixoto, Nathan Mattes e \\Pedro Scholz Soares}
\institute{Instituto de Informática --- UFRGS\\\texttt{elpeixoto@inf.ufrgs.br\\nmattes@inf.ufrgs.br\\pedro.soares@inf.ufrgs.br\\}}

\begin{document}

% Command to create title page
\InfTitlePage

% 1D Heat equation


\section{Detalhamento}

\frame{
    \frametitle{Descrição}

    \begin{itemize}
        \item O objeto de estudo deste trabalho é a solução numérica da equação do
         calor em uma, duas e três dimensões. 
		\item Essa equação descreve a forma como o calor se espalha por um objeto. 
        Será implementado algoritmos em duas linguagens de programação distintas, 
        julia e python. 
        \item A justificativa das linguagens vem do fato de python ser uma linguagem 
        consistente para uso científico devido suas bibliotecas e ser amplamente
        popular, por outro lado julia apresenta ótimos resultados em cálculo numérico.
    \end{itemize}
}


\frame{
    \frametitle{Descrição}
    \begin{figure}[h!]
    \centering
        \includegraphics[width=0.8\textwidth]{../../imagensEtapa2/OrgranogramaExperimentos.jpeg}
        \caption{Orgranograma dos Experimentos}
    \end{figure}
}

\section{Detalhamento}

\frame{
    \frametitle{Método de Coleta de Dados}

    \begin{itemize}
        \item Usaremos medição como método de análise.
        \item São algoritmos de implementação simples.
        \item Podemos analisar dados obtidos de forma empírica.
    \end{itemize}
}

\frame{
    \frametitle{Método de Coleta de Dados}
    
    \begin{itemize}
        \item Julia utiliza compilação just-in-time (JIT): 
        \begin{itemize}
            \item A primeira execução de um código em Julia pode ser mais lenta devido à necessidade de compilar o código antes de executá-lo.
            \item Em execuções subsequentes, o código já está compilado, o que resulta em tempos de execução significativamente mais rápidos.
        \end{itemize}
        \item Python é uma linguagem interpretada:
        \begin{itemize}
            \item Cada vez que um script Python é executado, o interpretador lê e interpreta o código linha por linha.
            \item Isso pode resultar em tempos de execução mais lentos, especialmente para tarefas computacionalmente intensivas.
        \end{itemize}
        \item Para garantir uma comparação justa, o tempo de compilação inicial do Julia será separado do tempo de execução.
        \item O foco da análise será o tempo de execução após a compilação inicial.

    \end{itemize}
}

\section{Detalhamento}

\frame{
    \frametitle{Resultados Preliminares}

    \begin{itemize}
        \item Resolução de EDP's já estudada pelos membros do grupo em outras disciplinas, através da abordagem analítica.
        \item Nenhum membro do grupo explorou a abordagem computacional da resolução da equação.
        \item Enxergamos esse trabalho como uma oportunidade de estudar a equação sob outra visão, nos possibilitando abordar conteúdos relativos a cálculo numérico.
    \end{itemize}
}

\section{Detalhamento}

\frame{
    \frametitle{Dificuldades e Soluções}

    \begin{itemize}
        \item Tempo de execução
        \begin{itemize}
            \item Coletar múltiplas amostras para fazer uma média e garantir robustez nos resultados
            \item Distinguir o tempo de compilação inicial do Julia do tempo de execução a fim de ter uma comparação mais justa com Python
        \end{itemize}
        \item Uso de memória
        \begin{itemize}
            \item Avaliar a eficiência da alocação de memória por ambas as linguagens
        \end{itemize}
    \end{itemize}
}

\section{Detalhamento}

\frame{
    \frametitle{Plano para Finalização e Cronograma Atualizado}
    \begin{tabular}{l r}
        Atividade & Prazo Máximo\\
        Organizar os scripts auxiliares & 11/10\\
        Ampliar a análise com novas comparações & 18/10\\
        Analise qualitativa dos resultados & 25/10\\
        Expecificação do hardware para o experimento & 01/11\\
        Coleta total dos dados & 15/11\\
        Análise final dos dados & 22/11\\
        Escrita do relatório final &  30/11\\
    \end{tabular}
}

\section*{}

\begin{frame}
    \frametitle{Obrigado!}
    \InfContacts
\end{frame}

\end{document}
